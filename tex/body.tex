% readme.tex -- a short example of how each Stata Journal insert should be
% organized.

\inserttype{article}
\author{Short article author list}{%
  Kerui Du\\School of Management, Xiamen University\\Xiamen, Fujian/China\\kerrydu@xmu.edu.cn

  \and
 Yonghui Zhang\\School of Economics, Renmin University of China\\Beijing/China\\ yonghui.zhang@hotmail.com

  \and
  Qiankun Zhou\\Department of Economics, Louisiana State University\\Baton Rouge/USA\\qzhou@lsu.edu


}

\title[Short toc article title]{Estimating partially linear functional-coefficient panel data models with Stata}
\maketitle

\begin{abstract}
	In this article, we describe Stata implementation of estimating partially linear functional-coefficient panel models with fixed effects proposed by \citet{yonghong2016semiparametric} and \citet{Zhang2018}. Three new commands xtplfc, ivxtplfc and xtdplfc are introduced and illustrated through Monte Carlo simulations to exemplify the effectiveness of these estimators.
	
	\keywords{xtplfc, ivxtplfc, xtdplfc, functional coefficients, fixed effects, sieve, spline}
\end{abstract}

% discussion of the Stata Journal document class.
%\input sj.tex
% discussion of the Stata Press LaTeX package for Stata output.
%\input stata.tex
\input first.tex
% discussion of the Stata Press LaTeX package for Stata output.
\input second.tex

\bibliographystyle{sj}
\bibliography{sj}

\begin{aboutauthors}
Kerui Du is an associate professor at School of Management, Xiamen University. His primary research interests are applied econometrics, energy and environmental economics.

Yonghui Zhang (corresponding author) is an assistant professor of Economics at School of Economics, Renmin University of China. His primary research interests are both theoretical and applied econometrics, with a focus on nonparametric econometrics.

Qiankun Zhou is an associate professor of Economics at the Department of Economics of Louisiana State University. His primary research interests are both theoretical and applied econometrics, with a focus on panel data econometrics.


\end{aboutauthors}

\endinput
